\documentclass[12pt]{article}

\usepackage[utf8]{inputenc}
\usepackage{sfmath}
\usepackage{amsmath}
\usepackage{graphics}

\usepackage{titling}
\pretitle{\begin{center}\Huge\bfseries}
\posttitle{\par\end{center}\vspace{0.5cm}}
\preauthor{\begin{center}\large}
\postauthor{\par\end{center}}
\predate{\begin{center}\large}
\postdate{\par\end{center}}

\usepackage{geometry}
\geometry{a4paper, margin=1in}

\title{Exoskeleton Robotic Hand With Wolf Claw Mechanism}

\author{Bryan Li, Yanpei Zhu, Nolan Yu}

\date{\today}


\begin{document}

\begin{titlepage}
	\huge
	\maketitle
	\begin{abstract}
		\Large
		This report presents the conceptual design, mathematical modeling,
		and mechanical analysis of an exoskeleton robotic hand integrated
		with a retractable wolf claw mechanism. Inspired by Marvel's
		Wolverine character, the project aims to explore a multi-functional
		prosthetic or assistive device solution that combines grasping
		capabilities with tool functionality.

		The report details two alternative wolf claw drive mechanisms
		(planetary gear system and compound gear train), analyzes their
		input-output relationships through mathematical models,
		and covers the kinematic modeling of the finger exoskeleton,
		wrist design, and overall system integration analysis, laying
		the foundation for the next stage of detailed design and manufacturing.
	\end{abstract}
\end{titlepage}

\tableofcontents

\pagebreak

\section{Introduction}
\subsection{Project Objectives and Description}
\subsection{Research Applications and Mechanical Choice}

\section{Conceptual Design}
\subsection{Finger Exoskeleton}
\subsection{Wolf Claw Mechanisms}
\subsection{Wrist Joint}

\section{Mathematical Modelling and Analysis}
\subsection{Driving Method and Transmission}
\subsection{Wolf Claw Mechanism - Version 1 (Planetary Gear)}
\subsection{Wolf Claw Mechanism - Version 2 (Compound Gear Train)}
\subsection{Fingers and Wrist Kinematics}

\section{Mechanical Analysis}
\subsection{Drive Method and Transmission}
\subsection{Wolf Claw Mechanism Comparison}
\subsection{Finger Motion Analysis}

\section{Conclusion and Future Work}
\subsection{Conclusion}
\subsection{Future Work}

\section{References}

\end{document}
